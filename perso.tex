

\documentclass{cv}

\setlength{\parskip}{0pt}

% ---------------------------------------------------------------------------------
% Set lengths.
% ---------------------------------------------------------------------------------

% The distance between the left margin and the start of the area that prints the 
% name of the section.
% |<left margin>|<---->|<company>
\newlength{\companySectionIndentLength}
\setlength{\companySectionIndentLength}{10pt}%

% ---------------------------------------------------------------------------------
% Set colors.
% ---------------------------------------------------------------------------------

% See https://www.sharelatex.com/learn/Using_colours_in_LaTeX
% See https://en.wikipedia.org/wiki/List_of_colors:_A%E2%80%93F
\definecolor{colorSection}{rgb}{0.97, 0.77, 0.56}
\definecolor{colorCompanySection}{rgb}{0.525, 0.917, 0.603}
\definecolor{colorVerticalTableLines}{rgb}{0.639, 0.639, 0.639}

% ---------------------------------------------------------------------------------
% Define the element that reprsents the start of a main section (languages, frame-
% -work).
% ---------------------------------------------------------------------------------

% Usage:
%
% \mainSectionSeparator{<section name>}
% or
% \mainSectionSeparator{<section name>}{yes}
%
% The later command call adds a vertical space after the section.

\newcommand{\mainSectionSeparator}[2]{
   \vspace{10pt}
   \begin{tcolorbox}[notitle,
                     nobeforeafter, % See https://mirror.hmc.edu/ctan/macros/latex/contrib/tcolorbox/tcolorbox.pdf
                     boxrule=0pt,
                     top=2pt,
                     bottom=2pt,
                     halign=center,
                     valign=center,
                     width=\textwidth,
                     colback={colorSection}]%
      #1%
   \end{tcolorbox}%
   \ifthenelse{\equal{#2}{yes}}{\vspace{10pt}}{}%
}%

% ---------------------------------------------------------------------------------
% Define the element that represents the start of a section that presents the 
% realisation for a company.
% ---------------------------------------------------------------------------------

\newcommand{\sectionCompany}[1]{ 

   \vspace{10pt}
   \hspace{\companySectionIndentLength}\begin{tcolorbox}[notitle,
                      nobeforeafter, % See https://mirror.hmc.edu/ctan/macros/latex/contrib/tcolorbox/tcolorbox.pdf
                      bottomrule=2pt,
                      toprule=0pt,
                      leftrule=0pt,
                      rightrule=0pt,
                      top=0pt,
                      bottom=0pt,
                      halign=left,
                      left=0pt,
                      valign=center,
                      width=\textwidth-\companySectionIndentLength]%
       #1%
   \end{tcolorbox}%
}%

% ---------------------------------------------------------------------------------
% Define commands for general text formatting.
% ---------------------------------------------------------------------------------

% This environment sets the required configuration for a quote.
\newenvironment{quotenv}{%
   \setmainfont[Ligatures=TeX]{Accanthis ADF Std Italic}
   \small
}

\newenvironment{technoenv}{%
   \setmainfont[Ligatures=TeX]{Ubuntu Mono}
}

\newcommand{\techno}[1]{%
   \begin{technoenv}%
   #1% 
   \end{technoenv}%
}

% This command declare a text as a quote.
\newcommand{\quoteit}[1]{%
   \begin{quotenv}%
   #1%
   \end{quotenv}%
   \vspace{10pt}
}

% This environment can be used to present a quoted text.
\newenvironment{packed_enum}{
   \begin{itemize}[topsep=0pt, itemsep=1pt, parsep=0pt, partopsep=0pt]%
}{\end{itemize}}

% This command prints a quoted text.
\newenvironment{packed_tabular}{
   \setlength{\tabcolsep}{0pt}
}

% Make sure that the content inserted within this environment is printed on one page.
% https://stackoverflow.com/questions/4003473/make-an-unbreakable-block-in-tex
\newenvironment{absolutelynopagebreak}{
  \par\nobreak\vfil\penalty0\vfilneg
  \vtop\bgroup
}{\par\xdef\tpd{\the\prevdepth}\egroup\prevdepth=\tpd}

% Format the name of a company.
\newcommand{\cpname}[1]{%
  \textbf{#1}%
}

% Format the name of a degree.
\newcommand{\school}[1]{%
  \textbf{#1}%
}

% ---------------------------------------------------------------------------------
% Define formats for texts that appear in the header.
% ---------------------------------------------------------------------------------

\DeclareTextFontCommand{\fontIdentity}{\setmainfont[Ligatures=TeX]{LMRoman6-Regular}}
\newcommand{\identity}[2]{\fontIdentity{#1 #2}}
\newcommand{\addressFirst}[1]{#1}
\newcommand{\addressSecond}[1]{#1}
\newcommand{\tel}[1]{#1}
\newcommand{\email}[2]{#1@#2}
\newcommand{\website}[1]{#1}
\newcommand{\drivingLicense}[1]{#1}

\newenvironment{headerEnv}{%
   \setmainfont[Ligatures=TeX]{LMRoman6-Regular}%
}

% ---------------------------------------------------------------------------------
% Define formats for texts that appear in tables.
% ---------------------------------------------------------------------------------

% This command defines the background color of the first line of the table (that is, the header).
\definecolor{colorHeader}{rgb}{0.77, 0.77, 0.56}

% This command applies the required style to the text that appears in the table header.
\newcommand{\tableHd}[1]{%
   \textbf{#1}
}%

% We define a column type for tabularx.
% This type centers the content of a cell.
% See https://tex.stackexchange.com/questions/89166/centering-in-tabularx-and-x-columns
\newcolumntype{C}{>{\centering\arraybackslash}X}

\newcommand{\realisationHeader}[2]{
   \vspace{10pt}
   \begin{tcolorbox}[notitle,
                     nobeforeafter, % See https://mirror.hmc.edu/ctan/macros/latex/contrib/tcolorbox/tcolorbox.pdf
                     bottomrule=0pt,
                     toprule=0pt,
                     leftrule=2pt,
                     rightrule=0pt,
                     top=0pt,
                     bottom=0pt,
                     halign=left,
                     left=0pt,
                     valign=center,
                     width=\textwidth,
                     colback=white]%
      \begin{packed_tabular}
         \begin{tabular}{lll}
            \textbf{Durée}        & ~ & #1 \\
            \textbf{Technologies} & ~ & #2 \\
         \end{tabular}%
      \end{packed_tabular}%
   \end{tcolorbox}%
}%

\newenvironment{realisationEnv}{
   \setlength{\parskip}{1em}
}


\begin{document}

   % ------------------------------------------------------------------------
   % HEADER
   % ------------------------------------------------------------------------

   \begin{headerEnv}%
      \identity{Denis}{BEURIVE} \newline
      \vspace{3pt} \leavevmode\newline
      % See https://tex.stackexchange.com/questions/128127/fix-column-width-with-tabularx
      \begin{tabularx}{\textwidth}{cc>{\raggedright\arraybackslash}X>{\raggedleft\arraybackslash}X}
          \faMapMarker & ~ & \addressFirst{12 av Anatole France} & 45 ans \\
                       & ~ & \addressSecond{92110 Clichy}        & Nationalité française \\
          \faPhone     & ~ & \tel{06 37 78 68 09}                & Célibataire, sans enfant \\
          \faAt        & ~ & \email{denis.beurive}{gmail.com}    & \\
          \faFirefox   & ~ & \website{http://www.beurive.com}    & \\
          \faCar       & ~ & Permis B                            & \\
      \end{tabularx}%
   \end{headerEnv}%
   

   % ------------------------------------------------------------------------
   % LANGAGES
   % ------------------------------------------------------------------------
   
   \mainSectionSeparator{Langages}{yes}

   \quoteit{La pratique d'un grand nombre de langages de programmation reflète mon expérience sur des problématiques et des
       environnements variés, caractérisés par des contraintes et des paradigmes également variés. Le choix du langage
       adapté est une des clés du succès.}

   \begin{tabularx}{\textwidth}{|X|X|X|C|}

       \arrayrulecolor{colorHeader}    

       \hline 
       \rowcolor{colorHeader}
       \tableHd{Langage} & \tableHd{Niveau} & \tableHd{Expérience} & \tableHd{Opérationnel} \\

       \arrayrulecolor{colorVerticalTableLines}    

       \hline
       PHP5/7        & Expert             & 20 ans & \faBatteryFull \\
       \hline
       Perl5         & Expert             & 20 ans & \faBatteryFull  \\
       \hline
       C             & Expert             & 10 ans & \faBatteryThreeQuarters \\
       \hline
       JavaScript    & Bonne connaissance & 3 ans  & \faBatteryThreeQuarters \\
       \hline
       GO            & Bonne connaissance & 1 ans  & \faBatteryThreeQuarters \\
       \hline
       Python        & Notions            & 6 mois & \faBatteryHalf \\
       \hline
       Java          & Bonne connaissance & 3 ans  & \faBatteryHalf \\
       \hline
       C++           & Ancien expert      & 4 ans  & \faBatteryQuarter \\
       \hline
       TCL           & Notions            & 6 mois & \faBatteryQuarter \\
       \hline
       ActionScript  & Ancien expert      & 1 an   & \faBatteryEmpty \\
       \hline
   \end{tabularx}%

   % ------------------------------------------------------------------------
   % FRAMEWORKS
   % ------------------------------------------------------------------------

   \mainSectionSeparator{Frameworks}{yes}

   \quoteit{Il y a plusieurs types de frameworks qui implémentent des "philosophies", et des "approches", diverses,
       à des niveaux différents. Il n'existe pas de frameworks "adaptés à tous les projets". Ils n'existent que des
       frameworks plus ou moins adaptés à des contextes d'utilisation. Le choix d'un framework adapté est une des clés
       de la réussite.}

   \begin{tabularx}{\textwidth}{|X|X|X|C|}

       \arrayrulecolor{colorHeader}    

       \hline 
       \rowcolor{colorHeader}
       \tableHd{Framework} & \tableHd{Type} & \tableHd{Langage} & \tableHd{Niveau actuel} \\

       \arrayrulecolor{colorVerticalTableLines}    

       \hline 
       Zend V1             & MVC            & PHP               & \faThumbsOUp \\
       \hline 
       Slim V3             & MVC            & PHP               & \faThumbsOUp \\
       \hline 
       Dancer              & MVC            & Perl5             & \faThumbsOUp \\
       \hline 
       JQuery              & DOM+Event      & JavaScript        & \faThumbsOUp \\
       \hline 
       ELGG                & MVC+Social     & PHP               & \faThumbsODown \\
       \hline 
       Adobe Flex 3        & FEWA*          & ActionScript      & \faThumbsODown \\
       \hline
       AngularJS V1.5      & FEWA*          & JavaScript        & \faThumbsODown \\
       \hline 
   \end{tabularx}
   \vspace{10pt}

   FEWA = Front-End Web Application

   % ------------------------------------------------------------------------
   % FUNCTIONAL
   % ------------------------------------------------------------------------

   \mainSectionSeparator{Compétences métier}{yes}

   % http://borntocode.fr/latex-customisation-de-listes-a-puces/
   \begin{packed_enum}
      \item Choix technologiques.
      \item Architecture logicielle.
      \item Architecture système.
      \item Conception de systèmes capables de monter en charge (scalable).
      \item Conception de systèmes robustes (qui conservent un comportement déterministe, même en dehors de la plage d’utilisation nominale pour laquelle ils sont dimensionnés).
      \item Conception de systèmes « transparents » (conçus, dès le départ, pour être supervisés).
      \item WEB
   \end{packed_enum}

   % ------------------------------------------------------------------------
   % PROFESSIONAL EXPERIENCE
   % ------------------------------------------------------------------------

   \newpage
   \mainSectionSeparator{Expérience professionnelle}{yes}

   \begin{packed_tabular}
      \begin{tabular}{lclclclll}
         09/ & 2016 & ~ & \multicolumn{2}{l}{à ce jour} & ~ & \cpname{Kertios}     & ~ &%
             Architecte + Ingénieur d'étude et développement \\
         09/ & 2013 & ~ & 07/ & 2015                    & ~ & \cpname{Ijenko}      & ~ &%
             Architecte + Ingénieur d'étude et développement \\
         01/ & 2013 & ~ & 07/ & 2013                    & ~ & \cpname{Halys}       & ~ &%
             Architecte + Ingénieur d'étude et développement \\
         06/ & 2009 & ~ & 11/ & 2012                    & ~ & \cpname{Paritel}     & ~ &%
             Architecte + Ingénieur d'étude et développement \\
         08/ & 2008 & ~ & 06/ & 2009                    & ~ & \cpname{FREE}        & ~ &%   
             Architecte + Ingénieur d'étude et développement \\
         06/ & 2005 & ~ & 08/ & 2008                    & ~ & \cpname{Alice ADSL}  & ~ &%
             Architecte + Ingénieur d'étude et développement \\
         09/ & 2001 & ~ & 06/ & 2005                    & ~ & \cpname{LibertySurf} & ~ &%
             Architecte + Ingénieur d'étude et développement \\
         03/ & 1999 & ~ & 09/ & 2000                    & ~ & \cpname{Cetia Inc}*  & ~ &%
             Architecte + Ingénieur d'étude et développement \\
         06/ & 1998 & ~ & 01/ & 1999                    & ~ & \cpname{STERIA}      & ~ &%
             Architecte + Ingénieur d'étude et développement \\
      \end{tabular}%
   \end{packed_tabular}%

   \vspace{10pt}

   \cpname{Cetia Inc} CETIA était le revendeur exclusif des produits Thales aux USA (Massachusetts). En 2000, Cetia a été englobé par Thales.

   % ------------------------------------------------------------------------
   % FORMATION
   % ------------------------------------------------------------------------

   \mainSectionSeparator{Formation}{yes}

   \begin{packed_tabular}
      \begin{tabular}{lclclcl}
         \school{DESS "Ingénierie des Systèmes Informatiques"} & ~ & BAC+5 & ~ & 1998 & ~ &%
              Université Pierre-et-Marie-Curie \\
         \school{Maîtrise de physique et applications}         & ~ & BAC+4 & ~ & 1997 & ~ &%
              Université Pierre-et-Marie-Curie \\
      \end{tabular}%
   \end{packed_tabular}%

   % ------------------------------------------------------------------------
   % PROJECTS
   % ------------------------------------------------------------------------

   \mainSectionSeparator{Projets}

   \sectionCompany{Kertios}

   \realisationHeader{6 mois}{\techno{Linux}, \techno{PHP7}, \techno{RabbitMQ}}

   \begin{realisationEnv}
      Développement d'un outil qui permet de construire un graphe de dépendances à partir d'une liste de formules mathématiques qui dépendent les
      unes des autres.

      Développement d'un moteur qui parallélise les calculs d'un ensemble de formules mathématiques, compte tenu d'un graphe de dépendances.
      Notre moteur est 20 fois plus rapide que le moteur d'origine. Indépendamment des données fournies au moteur (les formules mathématiques),
      le potentiel d'accélération est virtuellement illimité. Dans la pratique, le potentiel d'accélération dépend des données fournies au moteur,
      car ces données (les formules mathématiques) déterminent le nombre moyen de calculs pouvant être exécutés en parallèle.
   \end{realisationEnv}

   \realisationHeader{18 mois}{\techno{AIX}, \techno{Linux}, \techno{PHP7}, \techno{Perl5}, \techno{JavaScript}, \techno{SQLite}, \techno{TCL-TK}, \techno{KSH}, \techno{BASH}, \techno{jQuery}, \techno{Chart.js}}

   \begin{realisationEnv}
      Assurer les migrations d'UNICENTER (Computer Associates) vers CONTROLM (BMC Software) et de PELICAN (Axway) vers CFT (Axway). OS : AIX 7.2
      et linux.

      Développement d'un outil permettant de convertir un jeu de configurations PELICAN en configurations CFT. Cet outil vérifie la cohérence des
      données de configuration sur l'ensemble de configurations présentées en entrée (vérifications croisées). il détecte également toutes les
      mauvaises pratiques. Technologies utilisées : Perl et SQLite.

      Développement d'un outil permettant de convertir une configuration CONTROLM (XML) afin de l'adapter à de nouveaux environnements fonctionnels.
      Cet outil vérifie la cohérence des données de configuration et détecte des problèmes spécifiques à la plateforme du client (boucles...).
      Technologies utilisées : PHP7, TCL-TK, SQLite.

      Développement d'un convertisseur à la volée de commandes PELICAN en commande CFT. Ce convertisseur remplace le client PELICAN sur les serveurs
      (AIX et Linux) : il permet la migration de millions de lignes de code KSH de PELICAN vers CFT sans modifier un seul script. Ce convertisseur
      peut être configuré pour définir des exceptions (certains appels PELICAN sont effectués via PELICAN), en fonctions de plusieurs critères.
      Technologies utilisées : KSH (99\%) et Perl (1\%). 

      Développement d'un convertisseur à la volée de commandes UNICENTER en commande CONTROLM. Ce convertisseur remplace le client UNICENTER sur
      les serveurs (AIX et Linux) : il permet la migration de millions de lignes de code KSH de UNICENTER vers CONTROLM sans modifier un seul script.
      Technologies utilisées : KSH (99\%) et Perl (1\%).

      Développement d'un analyseur de LOG PELICAN destiné à produire une cartographie des flux sur la plateforme du client. Cet outil peut être
      couplé au convertisseur de configuration PELICAN en configuration CFT afin de détecter des configurations non utilisées. Technologies
      utilisées : Perl et SQLite.

      Développement d'un outil de supervision pour IBM Tivoli Storage Manager (TSM). Cet outil injecte les données extraites des LOG TSM dans une
      base de données SQLite et génère des rapports journaliers au format HTML (qui peuvent être consultés via HTTP). Technologies utilisées :
      Perl5, JavaScript, JQuery, Chart.js.

   \end{realisationEnv}

\end{document}
