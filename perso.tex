

\documentclass{cv}

% \usepackage{showframe}
\renewcommand{\enframe}{yes}

% See https://www.sharelatex.com/learn/Using_colours_in_LaTeX
% See https://en.wikipedia.org/wiki/List_of_colors:_A%E2%80%93F
\definecolor{colorSection}{rgb}{0.97, 0.77, 0.56}

% Define the element that reprsents the start of a section.
\newcommand{\sectionSeparator}[1]{
   \vspace{10pt}
   \begin{tcolorbox}[notitle,
                     boxrule=0pt,
                     top=2pt,
                     bottom=2pt,
                     halign=center,
                     valign=center,
                     width=\textwidth,
                     colback={colorSection}]
      #1
   \end{tcolorbox}
   \vspace{10pt}
}

% This environment sets the required configuration for a quote.
\newenvironment{quotenv}{%
   \setmainfont[Ligatures=TeX]{Accanthis ADF Std Italic}
   \small
}

% This command declare a text as a quote.
\newcommand{\quoteit}[1]{
   \begin{quotenv}
   #1
   \end{quotenv}
   \vspace{10pt}
}

\newenvironment{packed_enum}{
   \begin{itemize}[topsep=0pt, itemsep=1pt, parsep=0pt, partopsep=0pt]%
   % \begin{itemize}[topsep=0pt, itemsep=1pt, parsep=0pt]%
}{\end{itemize}}

\newenvironment{packed_tabular}{
   \setlength{\tabcolsep}{0pt}
}


\newenvironment{absolutelynopagebreak}{
  \par\nobreak\vfil\penalty0\vfilneg
  \vtop\bgroup
}{\par\xdef\tpd{\the\prevdepth}\egroup\prevdepth=\tpd}


% ---------------------------------------------------------------------------
% HEADER
% ---------------------------------------------------------------------------

\DeclareTextFontCommand{\fontIdentity}{\setmainfont[Ligatures=TeX]{LMRoman6-Regular}}
\newcommand{\identity}[2]{\fontIdentity{#1 #2}}
\newcommand{\addressFirst}[1]{#1}
\newcommand{\addressSecond}[1]{#1}
\newcommand{\tel}[1]{#1}
\newcommand{\email}[2]{#1@#2}
\newcommand{\website}[1]{#1}
\newcommand{\drivingLicense}[1]{#1}

\newenvironment{headerEnv}{%
   \setmainfont[Ligatures=TeX]{LMRoman6-Regular}%
}

% ---------------------------------------------------------------------------
% TABLES
% ---------------------------------------------------------------------------

% This command defines the background color of the first line of the table (that is, the header).
\definecolor{colorHeader}{rgb}{0.77, 0.77, 0.56}

% This command applies the required style to the text that appears in the table header.
\newcommand{\tableHd}[1]{%
   \textbf{#1}
}%

% We define a column type for tabularx.
% This type centers the content of a cell.
% See https://tex.stackexchange.com/questions/89166/centering-in-tabularx-and-x-columns
\newcolumntype{C}{>{\centering\arraybackslash}X}






\begin{document}

   % ------------------------------------------------------------------------
   % HEADER
   % ------------------------------------------------------------------------

   \begin{headerEnv}%
      \identity{Denis}{BEURIVE} \newline
      \vspace{3pt} \leavevmode\newline
      % See https://tex.stackexchange.com/questions/128127/fix-column-width-with-tabularx
      \begin{tabularx}{\textwidth}{cc>{\raggedright\arraybackslash}X>{\raggedleft\arraybackslash}X}
          \faMapMarker & ~ & \addressFirst{12 av Anatole France} & 45 ans \\
                       & ~ & \addressSecond{92110 Clichy}        & Nationalité française \\
          \faPhone     & ~ & \tel{06 37 78 68 09}                & Célibataire, sans enfant \\
          \faAt        & ~ & \email{denis.beurive}{gmail.com}    & \\
          \faFirefox   & ~ & \website{http://www.beurive.com}    & \\
          \faCar       & ~ & Permis B                            & \\
      \end{tabularx}%
   \end{headerEnv}%
   

   % ------------------------------------------------------------------------
   % LANGAGES
   % ------------------------------------------------------------------------
   
   \sectionSeparator{Langages}

   \quoteit{La pratique d'un grand nombre de langages de programmation reflète mon expérience sur des problématiques et des
       environnements variés, caractérisés par des contraintes et des paradigmes également variés. Le choix du langage
       adapté est une des clés du succès.}

   \begin{tabularx}{\textwidth}{|X|X|X|C|}
       \hline 
       \rowcolor{colorHeader}
       \tableHd{Langage} & \tableHd{Niveau} & \tableHd{Expérience} & \tableHd{Opérationnel} \\
       \hline 
       PHP5/7        & Expert             & 20 ans & \faBatteryFull \\
       \hline
       Perl5         & Expert             & 20 ans & \faBatteryFull  \\
       \hline
       C             & Expert             & 10 ans & \faBatteryThreeQuarters \\
       \hline
       JavaScript    & Bonne connaissance & 3 ans  & \faBatteryThreeQuarters \\
       \hline
       GO            & Bonne connaissance & 1 ans  & \faBatteryThreeQuarters \\
       \hline
       Python        & Notions            & 6 mois & \faBatteryHalf \\
       \hline
       Java          & Bonne connaissance & 3 ans  & \faBatteryHalf \\
       \hline
       C++           & Ancien expert      & 4 ans  & \faBatteryQuarter \\
       \hline
       TCL           & Notions            & 6 mois & \faBatteryQuarter \\
       \hline
       ActionScript  & Ancien expert      & 1 an   & \faBatteryEmpty \\
       \hline
   \end{tabularx}%

   % ------------------------------------------------------------------------
   % FRAMEWORKS
   % ------------------------------------------------------------------------

   \sectionSeparator{Frameworks}

   \quoteit{Il y a plusieurs types de frameworks qui implémentent des "philosophies", et des "approches", diverses,
       à des niveaux différents. Il n'existe pas de frameworks "adaptés à tous les projets". Ils n'existent que des
       frameworks plus ou moins adaptés à des contextes d'utilisation. Le choix d'un framework adapté est une des clés
       de la réussite.}

   \begin{tabularx}{\textwidth}{|X|X|X|C|}
       \hline 
       \rowcolor{colorHeader}
       \tableHd{Framework} & \tableHd{Type} & \tableHd{Langage} & \tableHd{Niveau actuel} \\
       \hline 
       Zend V1             & MVC            & PHP               & \faThumbsOUp \\
       \hline 
       Slim V3             & MVC            & PHP               & \faThumbsOUp \\
       \hline 
       Dancer              & MVC            & Perl5             & \faThumbsOUp \\
       \hline 
       JQuery              & DOM+Event      & JavaScript        & \faThumbsOUp \\
       \hline 
       ELGG                & MVC+Social     & PHP               & \faThumbsODown \\
       \hline 
       Adobe Flex 3        & FEWA*          & ActionScript      & \faThumbsODown \\
       \hline
       AngularJS V1.5      & FEWA*          & JavaScript        & \faThumbsODown \\
       \hline 
   \end{tabularx}
   \vspace{10pt}

   FEWA = Front-End Web Application

   % ------------------------------------------------------------------------
   % FUNCTIONAL
   % ------------------------------------------------------------------------

   \sectionSeparator{Compétences métier}

   % http://borntocode.fr/latex-customisation-de-listes-a-puces/
   \begin{packed_enum}
      \item Choix technologiques.
      \item Architecture logicielle.
      \item Architecture système.
      \item Conception de systèmes capables de monter en charge (scalable).
      \item Conception de systèmes robustes (qui conservent un comportement déterministe, même en dehors de la plage d’utilisation nominale pour laquelle ils sont dimensionnés).
      \item Conception de systèmes « transparents » (conçus, dès le départ, pour être supervisés).
      \item WEB
   \end{packed_enum}

   % ------------------------------------------------------------------------
   % PROFESSIONAL EXPERIENCE
   % ------------------------------------------------------------------------

   % https://stackoverflow.com/questions/4003473/make-an-unbreakable-block-in-tex
   \begin{absolutelynopagebreak}
      \sectionSeparator{Expérience professionnelle}

      \begin{packed_tabular}
         \begin{tabular}{lclclclll}
             09/ & 2016 & ~ & \multicolumn{2}{l}{à ce jour} & :~ & \textbf{Kertios} & ~ &%
                 Architecte + Ingénieur d'étude et développement \\
             09/ & 2013 & ~ & 07/ & 2015                    & :~ & \textbf{Ijenko} & ~ &%
                 Architecte + Ingénieur d'étude et développement \\
             09/ & 2013 & ~ & 07/ & 2015                    & :~ & \textbf{Ijenko} & ~ &%
                 Architecte + Ingénieur d'étude et développement \\
             09/ & 2013 & ~ & 07/ & 2015                    & :~ & \textbf{Ijenko} & ~ &%
                 Architecte + Ingénieur d'étude et développement \\
         \end{tabular}%
      \end{packed_tabular}%
   \end{absolutelynopagebreak}%



\end{document}
