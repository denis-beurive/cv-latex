
\documentclass{cv}

\setlength{\parskip}{0pt}

% ---------------------------------------------------------------------------------
% Set lengths.
%
% - hlengthCompanySection: this length is used within the section that
%   presents the realisations per company. It is the distance between the left
%   margin and the start of the area that prints the name of the company.
% - vlengthSectionTitleContent: the distance that separates the bottom of the
%   element that marks the start of a section and the (first) element that forms
%   the content of the section.
%
% Note:
%
% - hlength: horizontal distance.
% - vlength: vertical distance.
% ---------------------------------------------------------------------------------

\newlength{\hlengthCompanySection}
\setlength{\hlengthCompanySection}{10pt}%

\newlength{\vlengthSectionTitleContent}
\setlength{\vlengthSectionTitleContent}{8pt}%

% ---------------------------------------------------------------------------------
% Set colors.
%
% - colorTableHeader: the background color of the first line of a table.
% - colorHorizontalTableLines: the color of the vertical lines within a table.
% - colorMainSection: the background color of the box that materialises the start
%   of a main section.
% - colorCompanySection: the background color of the box that materialises the
%   start of the section that presents the realisations for a given company.
%
% See https://www.sharelatex.com/learn/Using_colours_in_LaTeX
% See https://en.wikipedia.org/wiki/List_of_colors:_A%E2%80%93F
% ---------------------------------------------------------------------------------

\definecolor{colorTableHeader}{rgb}{0.77, 0.77, 0.56}
\definecolor{colorHorizontalTableLines}{rgb}{0.639, 0.639, 0.639}
\definecolor{colorMainSection}{rgb}{0.97, 0.77, 0.56}
\definecolor{colorCompanySection}{rgb}{0.525, 0.917, 0.603}
\definecolor{colorBackTechnoName}{rgb}{1, 1, 0.102}

% ---------------------------------------------------------------------------------
% Set fonts.
%
% - fontIdentityDef: the font that is used to render the identity (the first name
%   and the last name).
% - fontPrivateDef: the font that is used to render personal informationi (phone,
%   address...).
% - fontCitationDef: the font used to render citations.
% - fontTechnoDef: the font used to render the name of a technological tool.
% ---------------------------------------------------------------------------------

\newcommand{\fontIdentityDef}{\setmainfont[Ligatures=TeX]{LMRoman6-Regular}}
\newcommand{\fontPrivateDef}{\setmainfont[Ligatures=TeX]{LMRoman6-Regular}}
\newcommand{\fontCitationDef}{\setmainfont[Ligatures=TeX]{Accanthis ADF Std Italic}\small}
\newcommand{\fontTechnoDef}{\setmainfont[Ligatures=TeX]{Ubuntu Mono}}

\DeclareTextFontCommand{\fontIdentity}{\fontIdentityDef}
\DeclareTextFontCommand{\fontPrivate}{\fontPrivateDef}
\DeclareTextFontCommand{\fontCitation}{\fontCitationDef}
\DeclareTextFontCommand{\fontTechno}{\fontTechnoDef}

% ---------------------------------------------------------------------------------
% Technical terms.
% ---------------------------------------------------------------------------------

\newtcbox{\xtechnoBox}{on line,
                       boxsep=0pt,
                       boxrule=1pt,
                       colback={colorBackTechnoName},
                       colframe=white,
                       left=1pt,
                       right=1pt,
                       top=1pt,
                       bottom=1pt}

\newcommand{\technoBox}[1]{\xtechnoBox{\fontTechno{#1}}}

%%% Languages

\newcommand{\xvPerl}[1]{\technoBox{Perl#1}}
\newcommand{\xvPHP}[1]{\fontTechno{PHP#1}}
\newcommand{\xvPython}[1]{\fontTechno{Python#1}}

\newcommand{\xGo}{\fontTechno{Go}}
\newcommand{\xSQL}{\fontTechno{SQL}}
\newcommand{\xJava}{\fontTechno{Java}}
\newcommand{\xJavaScript}{\fontTechno{JavaScript}}
\newcommand{\xC}{\fontTechno{C}}
\newcommand{\xCpp}{\fontTechno{C++}}
\newcommand{\xTCL}{\fontTechno{TCL}}
\newcommand{\xTCLTK}{\fontTechno{Tcl/Tk}}
\newcommand{\xLaTeX}{\fontTechno{LaTeX}}

%%% Web

\newcommand{\xHTML}{\fontTechno{HTML}}
\newcommand{\xCSS}{\fontTechno{CSS}}
\newcommand{\xAjax}{\fontTechno{Ajax}}

%%% Sofwares

\newcommand{\xIbmTSM}{\technoBox{IBM Tivoli Storage Manager}}
\newcommand{\xTSM}{\technoBox{TSM}}
\newcommand{\xGit}{\fontTechno{Git}}
\newcommand{\xSVN}{\fontTechno{SVN}}
\newcommand{\xCVS}{\fontTechno{CVS}}
\newcommand{\xMySql}{\fontTechno{MySql}}
\newcommand{\xSQLite}{\fontTechno{SQLite}}
\newcommand{\xRabbitMQ}{\fontTechno{RabbitMQ}}
\newcommand{\xControlM}{\fontTechno{Control-M}}
\newcommand{\xUNICENTER}{\fontTechno{UNICENTER}}
\newcommand{\xCFT}{\fontTechno{CFT}}
\newcommand{\xPELICAN}{\fontTechno{Inter.Pel(Pelican)}}

%%% Framworks

\newcommand{\xvAngularJS}[1]{\fontTechno{AngularJS V#1}}
\newcommand{\xvZF}[1]{\fontTechno{Zend Framzwork V#1}}
\newcommand{\xvSlim}[1]{\fontTechno{Slim V#1}}

\newcommand{\xJQuery}{\fontTechno{jQuery}}
\newcommand{\xDancer}{\fontTechno{Dancer}}
\newcommand{\xFlex}{\fontTechno{Flex 3}}
\newcommand{\xELGG}{\fontTechno{ELGG}}

% Data

\newcommand{\xXML}{\technoBox{XML}}
\newcommand{\xJSON}{\technoBox{JSON}}

% ---------------------------------------------------------------------------------
% Define the elements that compose a section.
%
% - The section title.
% - The subsection title that displays the name of a company.
% - The subsection header that marks the start of a realisation. This header shows
%   the duration of the realisation and the tools used.
% - A (sub)section component. Please note that a (sub)section may be made of 
%   several components.
%
% Notes:
%
% - each element is responsible for applying the required vertical space that
%   separates it from the element that precedes it.
% - Each element is responsible for breaking the horizontal (or "paragraphe") mode
%   at its bottom.
% ---------------------------------------------------------------------------------

% Define the zone that prints the title of the section.
\newcommand{\mainSectionSeparator}[1]{%
   % Apply the required vertical space that separates it from the element that precedes it.
   \vspace{10pt}\par%
   \begin{tcolorbox}[notitle,
                     nobeforeafter, % See https://mirror.hmc.edu/ctan/macros/latex/contrib/tcolorbox/tcolorbox.pdf
                     boxrule=0pt,
                     top=1pt,
                     bottom=1pt,
                     halign=center,
                     valign=center,
                     width=\textwidth,
                     colback={colorMainSection}]%
      #1%
   \end{tcolorbox}%
   % Break the horizontal mode
   \par% 
}%

% Define the environment that applies to a section component.
% Please note that a section may be made of several components.
\newenvironment{sectionComponent}{%
   % Apply the required vertical space that separates it from the element that precedes it.
   \vspace{\vlengthSectionTitleContent}\par%
}{%
   % Break the horizontal mode
   \par%
}%

% Print the header that marks a eealisation.
% The first parameter represents the duration of the realisation.
% The second parameter represents the list of tools used.
\newcommand{\realisationHeader}[2]{%
   % Apply the required vertical space that separates it from the element that precedes it.
   \vspace{10pt}\par%
   \begin{tcolorbox}[notitle,
                     nobeforeafter, % See https://mirror.hmc.edu/ctan/macros/latex/contrib/tcolorbox/tcolorbox.pdf
                     bottomrule=0pt,
                     toprule=0pt,
                     leftrule=2pt,
                     rightrule=0pt,
                     top=0pt,
                     bottom=0pt,
                     halign=left,
                     left=0pt,
                     valign=center,
                     width=\textwidth,
                     colback=white]%
      \begin{packed_tabular}%
         \begin{tabular}{lll}%
            \textbf{Durée}        & ~ & #1 \\
            \textbf{Technologies} & ~ & #2 \\
         \end{tabular}%
      \end{packed_tabular}%
   \end{tcolorbox}%
   % Break the horizontal mode
   \par%
}%

% Define the element that represents the start of a sub-section that presents the realisation for a company.
\newcommand{\sectionCompany}[1]{% 
   % Apply the required vertical space that separates it from the element that precedes it.
   \vspace{\vlengthSectionTitleContent}\par%
   \hspace{\hlengthCompanySection}\begin{tcolorbox}[notitle,
                      nobeforeafter, % See https://mirror.hmc.edu/ctan/macros/latex/contrib/tcolorbox/tcolorbox.pdf
                      bottomrule=2pt,
                      toprule=0pt,
                      leftrule=0pt,
                      rightrule=0pt,
                      top=0pt,
                      bottom=0pt,
                      halign=left,
                      left=0pt,
                      valign=center,
                      colback={colorCompanySection},
                      width=\textwidth-\hlengthCompanySection]%
       #1%
   \end{tcolorbox}%
   % Break the horizontal mode
   \par%
}%

% ---------------------------------------------------------------------------------
% Define commands for general text formatting.
% ---------------------------------------------------------------------------------

% This environment sets the required configuration for a synthesis.
% Please note that this environment is used exclusively within the command \synthesis.
\newenvironment{synthesisEnv}{%
   \fontCitationDef
}{}

% This command creates a synthesis.
\newcommand{\synthesis}[1]{%
   \begin{synthesisEnv}%
   #1%
   \end{synthesisEnv}\par%
}

% This environment sets the required configuration for the rendering of a technical term.
% Please note that this environment is used exclusively within the command \techno.
\newenvironment{technoEnv}{%
   \fontTechnoDef
}{}

% This command prints a technical term.
% Ex: \techno{PHP}
\newcommand{\techno}[1]{%
   \begin{technoEnv}%
   #1% 
   \end{technoEnv}\par%
}

% This environment prints a "packed list of items".
\newenvironment{packed_enum}{
   \begin{itemize}[topsep=0pt, itemsep=1pt, parsep=0pt, partopsep=0pt]%
}{\end{itemize}}

% This command prints a "packed table".
\newenvironment{packed_tabular}{
   \setlength{\tabcolsep}{0pt}
}{}

% Make sure that the content inserted within this environment is printed on one page.
% https://stackoverflow.com/questions/4003473/make-an-unbreakable-block-in-tex
\newenvironment{absolutelynopagebreak}{
   \par\nobreak\vfil\penalty0\vfilneg
   \vtop\bgroup
}{%
   \par\xdef\tpd{\the\prevdepth}\egroup\prevdepth=\tpd%
}

% Format the name of a company.
% Ex: \cpname{Google}
\newcommand{\cpname}[1]{%
  \textbf{#1}%
}

% Format the name of a degree.
% Ex: \diploma[thesis}
\newcommand{\diploma}[1]{%
  \textbf{#1}%
}

% ---------------------------------------------------------------------------------
% Define formats for texts that appear in the header.
% ---------------------------------------------------------------------------------

% Print the first name and the last name.
\newcommand{\identity}[2]{\fontIdentity{#1 #2}}

% Print the first line of the address.
\newcommand{\addressFirst}[1]{#1}

% Print the second line of the address.
\newcommand{\addressSecond}[1]{#1}

% Print the telephone number.
\newcommand{\tel}[1]{#1}

% Print the email address.
\newcommand{\email}[2]{#1@#2}

% Print the URL to the website.
\newcommand{\website}[1]{#1}

% Print the type of the driving licence.
\newcommand{\drivingLicense}[1]{#1}

% Define the environment used to render the resume header.
\newenvironment{headerEnv}{%
   \fontPrivateDef%
}{}

% ---------------------------------------------------------------------------------
% Define formats for texts that appear in tables.
% ---------------------------------------------------------------------------------

% This command applies the required style to the text that appears in the table header.
\newcommand{\tableHd}[1]{%
   \textbf{#1}
}%

% We define a column type for tabularx.
% This type centers the content of a cell.
% See https://tex.stackexchange.com/questions/89166/centering-in-tabularx-and-x-columns
\newcolumntype{C}{>{\centering\arraybackslash}X}

% This environment applies to the description of a realisation.
\newenvironment{realisationEnv}{
   \setlength{\parskip}{1em}
}{%
   \par%
}


\begin{document}

   % ------------------------------------------------------------------------
   % HEADER
   % ------------------------------------------------------------------------

   \begin{headerEnv}%
      \identity{Denis}{BEURIVE} \newline
      \vspace{3pt} \leavevmode\newline
      % See https://tex.stackexchange.com/questions/128127/fix-column-width-with-tabularx
      \begin{tabularx}{\textwidth}{cc>{\raggedright\arraybackslash}X>{\raggedleft\arraybackslash}X}
          \faMapMarker & ~ & \addressFirst{12 av Anatole France} & 45 ans \\
                       & ~ & \addressSecond{92110 Clichy}        & Nationalité française \\
          \faPhone     & ~ & \tel{06 37 78 68 09}                & Célibataire, sans enfant \\
          \faAt        & ~ & \email{denis.beurive}{gmail.com}    & \\
          \faFirefox   & ~ & \website{http://www.beurive.com}    & \\
          \faCar       & ~ & Permis B                            & \\
      \end{tabularx}%
   \end{headerEnv}%

   % ------------------------------------------------------------------------
   % LANGAGES
   % ------------------------------------------------------------------------

   \mainSectionSeparator{Langages}

   \begin{sectionComponent}
      \synthesis{La pratique d'un grand nombre de langages de programmation reflète mon expérience sur des problématiques et des
          environnements variés, caractérisés par des contraintes et des paradigmes également variés. Le choix du langage
          adapté est une des clés du succès.}
   \end{sectionComponent}
   
   \begin{sectionComponent}
      \begin{tabularx}{\textwidth}{|X|X|X|C|}
   
          \arrayrulecolor{colorTableHeader}    
   
          \hline 
          \rowcolor{colorTableHeader}
          \tableHd{Langage} & \tableHd{Niveau} & \tableHd{Expérience} & \tableHd{Opérationnel} \\
   
          \arrayrulecolor{colorHorizontalTableLines}    
   
          \hline
          PHP5/7        & Expert             & 20 ans & \faBatteryFull \\
          \hline
          Perl5         & Expert             & 20 ans & \faBatteryFull  \\
          \hline
          C             & Expert             & 10 ans & \faBatteryThreeQuarters \\
          \hline
          JavaScript    & Bonne connaissance & 3 ans  & \faBatteryThreeQuarters \\
          \hline
          GO            & Bonne connaissance & 1 ans  & \faBatteryThreeQuarters \\
          \hline
          Python        & Notions            & 6 mois & \faBatteryHalf \\
          \hline
          Java          & Bonne connaissance & 3 ans  & \faBatteryHalf \\
          \hline
          C++           & Ancien expert      & 4 ans  & \faBatteryQuarter \\
          \hline
          TCL           & Notions            & 6 mois & \faBatteryQuarter \\
          \hline
          ActionScript  & Ancien expert      & 1 an   & \faBatteryEmpty \\
          \hline
      \end{tabularx}
   \end{sectionComponent}

   % ------------------------------------------------------------------------
   % FRAMEWORKS
   % ------------------------------------------------------------------------

   \mainSectionSeparator{Frameworks}

   \begin{sectionComponent}
      \synthesis{Il y a plusieurs types de frameworks qui implémentent des "philosophies", et des "approches", diverses,
          à des niveaux différents. Il n'existe pas de frameworks "adaptés à tous les projets". Ils n'existent que des
          frameworks plus ou moins adaptés à des contextes d'utilisation. Le choix d'un framework adapté est une des
          clés de la réussite.}
   \end{sectionComponent}
   
   \begin{sectionComponent}
      \begin{tabularx}{\textwidth}{|X|X|X|C|}
   
          \arrayrulecolor{colorTableHeader}    
   
          \hline 
          \rowcolor{colorTableHeader}
          \tableHd{Framework} & \tableHd{Type} & \tableHd{Langage} & \tableHd{Niveau actuel} \\
   
          \arrayrulecolor{colorHorizontalTableLines}    
   
          \hline 
          Zend V1             & MVC            & PHP               & \faThumbsOUp \\
          \hline 
          Slim V3             & MVC            & PHP               & \faThumbsOUp \\
          \hline 
          Dancer              & MVC            & Perl5             & \faThumbsOUp \\
          \hline 
          JQuery              & DOM+Event      & JavaScript        & \faThumbsOUp \\
          \hline 
          ELGG                & MVC+Social     & PHP               & \faThumbsODown \\
          \hline 
          Adobe Flex 3        & MVC            & ActionScript      & \faThumbsODown \\
          \hline
          AngularJS V1.5      & MVC            & JavaScript        & \faThumbsODown \\
          \hline 
      \end{tabularx}
   \end{sectionComponent}

   % ------------------------------------------------------------------------
   % FUNCTIONAL
   % ------------------------------------------------------------------------

   \mainSectionSeparator{Compétences métier}
   \begin{sectionComponent}
      % http://borntocode.fr/latex-customisation-de-listes-a-puces/
      \begin{packed_enum}
         \item Choix technologiques.
         \item Architecture logicielle.
         \item Architecture système.
         \item Conception de systèmes capables de monter en charge (scalable).
         \item Conception de systèmes robustes (qui conservent un comportement déterministe, même en dehors de la plage d’utilisation nominale pour laquelle ils sont dimensionnés).
         \item Conception de systèmes « transparents » (conçus, dès le départ, pour être supervisés).
         \item WEB
      \end{packed_enum}
   \end{sectionComponent}

   % ------------------------------------------------------------------------
   % PROFESSIONAL EXPERIENCE
   % ------------------------------------------------------------------------

   \newpage
   \mainSectionSeparator{Expérience professionnelle}

   \begin{sectionComponent}
      \begin{packed_tabular}
         \begin{tabular}{lclclclll}
            09/ & 2016 & ~ & \multicolumn{2}{l}{à ce jour} & ~ & \cpname{Kertios}     & ~ &%
                Architecte + Ingénieur d'étude et développement \\
            09/ & 2013 & ~ & 07/ & 2015                    & ~ & \cpname{Ijenko}      & ~ &%
                Architecte + Ingénieur d'étude et développement \\
            01/ & 2013 & ~ & 07/ & 2013                    & ~ & \cpname{Halys}       & ~ &%
                Architecte + Ingénieur d'étude et développement \\
            06/ & 2009 & ~ & 11/ & 2012                    & ~ & \cpname{Paritel}     & ~ &%
                Architecte + Ingénieur d'étude et développement \\
            08/ & 2008 & ~ & 06/ & 2009                    & ~ & \cpname{FREE}        & ~ &%   
                Architecte + Ingénieur d'étude et développement \\
            06/ & 2005 & ~ & 08/ & 2008                    & ~ & \cpname{Alice ADSL}  & ~ &%
                Architecte + Ingénieur d'étude et développement \\
            09/ & 2001 & ~ & 06/ & 2005                    & ~ & \cpname{LibertySurf} & ~ &%
                Architecte + Ingénieur d'étude et développement \\
            03/ & 1999 & ~ & 09/ & 2000                    & ~ & \cpname{Cetia Inc}   & ~ &%
                Architecte + Ingénieur d'étude et développement \\
            06/ & 1998 & ~ & 01/ & 1999                    & ~ & \cpname{STERIA}      & ~ &%
                Architecte + Ingénieur d'étude et développement \\
         \end{tabular}%
      \end{packed_tabular}%
      \par\vspace{10pt}\par
   
      \cpname{Cetia Inc} CETIA était le revendeur exclusif des produits Thales aux USA (Massachusetts). En 2000, Cetia a été englobé par Thales.
   \end{sectionComponent}

   % ------------------------------------------------------------------------
   % FORMATION
   % ------------------------------------------------------------------------

   \mainSectionSeparator{Formation}

   \begin{sectionComponent}
      \begin{packed_tabular}
         \begin{tabular}{lclclcl}
            \diploma{DESS "Ingénierie des Systèmes Informatiques"} & ~ & BAC+5 & ~ & 1998 & ~ &%
                 Université Pierre-et-Marie-Curie \\
            \diploma{Maîtrise de physique et applications}         & ~ & BAC+4 & ~ & 1997 & ~ &%
                 Université Pierre-et-Marie-Curie \\
         \end{tabular}%
      \end{packed_tabular}%
   \end{sectionComponent}

   % ------------------------------------------------------------------------
   % PROJECTS
   % ------------------------------------------------------------------------

   \mainSectionSeparator{Projets}

   \begin{sectionComponent}
      \sectionCompany{Kertios}
   
      \realisationHeader{6 mois}{\techno{Linux}, \techno{PHP7}, \techno{RabbitMQ}}
   
      \begin{realisationEnv}
         Développement d'un outil qui permet de construire un graphe de dépendances à partir d'une liste de formules mathématiques qui dépendent les
         unes des autres.
   
         Développement d'un moteur qui parallélise les calculs d'un ensemble de formules mathématiques, compte tenu d'un graphe de dépendances.
         Notre moteur est 20 fois plus rapide que le moteur d'origine. Indépendamment des données fournies au moteur (les formules mathématiques),
         le potentiel d'accélération est virtuellement illimité. Dans la pratique, le potentiel d'accélération dépend des données fournies au moteur,
         car ces données (les formules mathématiques) déterminent le nombre moyen de calculs pouvant être exécutés en parallèle.
      \end{realisationEnv}
   
      \realisationHeader{18 mois}{\techno{AIX}, \techno{Linux}, \techno{PHP7}, \techno{Perl5}, \techno{JavaScript}, \techno{SQLite}, \techno{TCL-TK}, \techno{KSH}, \techno{BASH}, \techno{jQuery}, \techno{Chart.js}}
   
      \begin{realisationEnv}
         Assurer les migrations d'UNICENTER (Computer Associates) vers CONTROLM (BMC Software) et de PELICAN (Axway) vers CFT (Axway). OS : AIX 7.2
         et linux.
   
         Développement d'un outil permettant de convertir un jeu de configurations PELICAN en configurations CFT. Cet outil vérifie la cohérence des
         données de configuration sur l'ensemble de configurations présentées en entrée (vérifications croisées). il détecte également toutes les
         mauvaises pratiques. Technologies utilisées : Perl et SQLite.
   
         Développement d'un outil permettant de convertir une configuration CONTROLM (XML) afin de l'adapter à de nouveaux environnements fonctionnels.
         Cet outil vérifie la cohérence des données de configuration et détecte des problèmes spécifiques à la plateforme du client (boucles...).
         Technologies utilisées : PHP7, TCL-TK, SQLite.
   
         Développement d'un convertisseur à la volée de commandes PELICAN en commande CFT. Ce convertisseur remplace le client PELICAN sur les serveurs
         (AIX et Linux) : il permet la migration de millions de lignes de code KSH de PELICAN vers CFT sans modifier un seul script. Ce convertisseur
         peut être configuré pour définir des exceptions (certains appels PELICAN sont effectués via PELICAN), en fonctions de plusieurs critères.
         Technologies utilisées : KSH (99\%) et Perl (1\%). 
   
         Développement d'un convertisseur à la volée de commandes UNICENTER en commande CONTROLM. Ce convertisseur remplace le client UNICENTER sur
         les serveurs (AIX et Linux) : il permet la migration de millions de lignes de code KSH de UNICENTER vers CONTROLM sans modifier un seul script.
         Technologies utilisées : KSH (99\%) et Perl (1\%).
   
         Développement d'un analyseur de LOG PELICAN destiné à produire une cartographie des flux sur la plateforme du client. Cet outil peut être
         couplé au convertisseur de configuration PELICAN en configuration CFT afin de détecter des configurations non utilisées. Technologies
         utilisées : Perl5 et SQLite.
   
         Développement d'un outil de supervision pour IBM Tivoli Storage Manager (TSM). Cet outil injecte les données extraites des LOG TSM dans une
         base de données SQLite et génère des rapports journaliers au format HTML (qui peuvent être consultés via HTTP). Technologies utilisées :
         Perl5, JavaScript, JQuery, Chart.js.
   
       \end{realisationEnv}
   \end{sectionComponent}

   \begin{sectionComponent}
      \sectionCompany{Ijenko}

      \realisationHeader{18 mois}{\techno{Linux}, \techno{PHP7}, \techno{RabbitMQ}}

      \begin{realisationEnv}
         \xvPerl{5} azerty
      \end{realisationEnv}

   \end{sectionComponent}

\end{document}
